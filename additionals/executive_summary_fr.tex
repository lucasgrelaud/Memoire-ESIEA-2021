%!TEX root = ../main.tex
\chapter*{Résumé Analytique}
Le sujet de la sécurisation des développements applicatifs est souvent une question épineuse pour les industries, le 
développement n'étant en réalité qu'un sous-produit permettant de répondre à un besoin technique.
\newline Cette question d'autant plus problématique plus lorsque ces mêmes industries cherchent à optimiser leurs systèmes d'informations 
en exploitant des technologies de conteneurisation et du \emph{Cloud computing}.

Dans ce mémoire de fin d'études, nous allons donc tenter de trouver des solutions répondant à cette problématique de 
sécurisation des chaînes de développement et de déploiement d'applications conteneurisées. Puis, dans un second temps, 
nous mettrons  en œuvre certaines de ces solutions sur différents portions de la chaîne de développement et la chaîne 
de déploiement et évaluerons l'efficacité de ces solutions.
\newline Ce travail démontreront d'ailleurs le besoin de collaboration entre les équipes de sécurité informatique,
d'architecture et d'infrastructure ainsi que des équipes de développement.

Nous conclurons ensuite sur un premier retour d'expérience et une démonstration des résultats obtenus grâce à 
l'implémentation de la sécurité dans les processus de développements applicatifs et dans la gestion des infrastructures
Kubernetes.
Enfin, nous évoquerons les possibles projets pouvant faire suite au travail réalisé dans ce mémoire, notamment au travers
de la formation des équipes.
