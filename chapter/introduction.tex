%!TEX root = ../main.tex
\chapter{Introduction}
Cette mission de fin d'étude clôture trois années d'apprentissage réalisées au sein de l'équipe de sécurité informatique 
du Groupe JCDecaux. Elle permet aussi de finaliser un travail de longue haleine visant à sécuriser et moderniser
la chaîne de production des applications et outils exploités par le groupe et ses filiales.

\section{Le Groupe JCDecaux}
JCDecaux est un groupe industriel international fondée en France en 1964 par Jean-Claude Decaux et spécialisé dans 
l'affichage publicitaire urbain. Principalement connue pour son mobilier urbain (Abribus, MUPI\footnote{Mobilier Urbain 
Pour l'Information - Panneau d'affichage de 2m² dont l'une des faces est réservée pour les collectivités.}, Senior
\footnote{Panneau d'affichage de 4m² en fixation murale ou aérienne.}), l'entreprise familiale a su se hisser en tant 
que leader de la communication extérieure et est reconnue comme étant N°1 mondiale dans ce domaine depuis 2001. 
\newline Son Chiffre d'Affaires pour l'année fiscale 2019 était de 3 890.2 M€.

La notoriété du Groupe JCDecaux s'explique par sa présence dans plus de 80 pays répartis sur tous les continents mais
aussi de par la qualité des services qu'il propose aux collectivités locales et à ses annonceurs.
\newline Cette recherche de l'excellence esthétique et technique pour ses produits fait partie intégrante des 
valeurs de l'entreprise. Ces valeurs sont une réelle fierté pour ses 12 300 collaborateurs et source d'innovation dans leurs
projets. 

\section{L'équipe Sécurité des Systèmes d'Informations}
\begin{wrapfigure}{R}{0.27\textwidth} 
    \centering \includegraphics[width=0.25\textwidth]{resources/img/jcd_pla_front.jpg}
    \centering Siège de Sainte Apolline
\end{wrapfigure}
En 2013, la \ac{DSI} Groupe s'est pourvu d'une équipe dédiée à la \ac{SSI}.
Le besoin d'une équipe dédié devenait en effet de plus en plus urgent suite à l'augmentation des ressources numérique
exploitées par le Groupe; tant pour la diffusion publicitaire programmatique que les services numériques aux 
collectivités (affichage, VLS).

L'équipe \ac{SSI} Groupe de JCDecaux est localisée sur le plateau de la \ac{DSI}, au siège de Sainte-Apolline, Plaisir.
Elle est actuellement composée de trois personnels internes (le \ac{RSSI}, un ingénieur sécurité, un apprenti) et est 
renforcée par trois personnels externes en prestation.
\newline Ces missions s'articulent autours de trois thèmes : la sécurité organisationnelle (Analyse de risque, 
politiques de sécurité, sensibilisation, \dots), la sécurité opérationnelle (SOC, conformité, veille phishing, \dots) et 
la qualification technique des systèmes et infrastructures.
\newline Spécificité de l'équipe \ac{SSI}, ses services sont offerts par le Groupe aux filiales afin de promouvoir la 
sécurité informatique dans ces structures.

\section{Publicité et développement applicatif}
Contraint par la complexité de son secteur d'activité, le Groupe JCDecaux à du se pourvoir en capacité de développement
logiciel et embarqué pour répondre à ses problématiques métier (\eg Booking, préparation, diffusion).
Ses équipes de développements sont réparties dans les différentes \ac{BU} des \ac{DSI} filiale et Groupe.

Depuis 2016, les équipes de développement travaillent au mise en place de technique agiles et à l'intégration d'une 
méthodologie DevOps dans leurs projets. Cet objectif, bientôt atteint, vise à accélérer les développements réalisés pour 
l'entreprise tout en augmentant la flexibilité des applications réalisées. 

Effet de bords bénéfique à ce changement de doctrine, on constate une augmentation de la fragmentation des applications
(séparation en micro-services et dissociation Frontend applicatif du Backend). Cette fragmentation permet en effet une
simplification des opérations de maintenances et favorise l'amélioration incrémentale des applications produites.

\section{Automatisation et conteneurisation du Si}
Conséquence de la modularisation de ses systèmes, le nombre de serveurs et machines virtuelles exploitées par les 
\ac{DSI} a drastiquement augmenté ces cinq dernières années.

Pour tenter de palier aux problématiques d'administration induites par cette augmentation, les équipes de la \ac{DSI} 
Groupe se sont pourvu en outils d'automatisation et méthodes de gestion pour optimiser leurs actions les plus récurrentes.
Grâce à l'amélioration de la \ac{CMP}, sa liaison avec la \ac{CMDB} et l'intégration d'Ansible \footnote{Ansible est un 
moteur d'automatisation du SI. Il permet l'automatisation de la configuration de système et du déploiement applicatif
grâce à des playbook et est initialement développé par RedHat sous licence GNU GPL v3.0}, les équipes opérationnelles 
de la DSI ont réussis à réduire la complexité de la gestion du \ac{SI} du Groupe.

Cependant, en 2019, un programme de qualification de la technologie d'orchestration de conteneur \ac{K8S} est monté 
conjointement entre la \ac{DSI} Groupe et la \ac{DSI} France. L'objectif, à terme, est de fournir aux équipes de 
développement la capacité de produire une application conteneurisée et directement déployable sur l'infrastructure.

Le programme vise à évaluer une architecture de cluster \ac{K8S} hébergé sur l'infrastructure de notre Cloud 
Provider \ac{AWS} et à l'intégration des outils de déploiement nécessaires. Il est actuellement sur le point de se 
terminer et la préparation d'une ouverture globale aux équipes des deux \ac{DSI} est en cours.

Une fois l'ouverture effective et l'analyse de l'efficacité de la plateforme réalisée pour les premiers mois, cette 
dernière sera rendu accessible aux filiales en faisant la demande.

\section{Problématiques de sécurité}
Comme pour tous les projets d'infrastructure informatique du Groupe, la plateforme \ac{K8S} doit respecter un ensemble de
règles et principes d'implémentation définies par les politiques du Groupe JCDecaux. La conformité de la plateforme aux
\emph{IT Security Policies} (\ie Politiques de sécurité informatique en français) est quant à elle un prérequis pour tout
déploiement en production. 
\newline En effet, toute non-conformité vis-à-vie d'une \emph{IT Security Policies} (\eg Réseau, Identité, Donnée, etc..)
pourrait entraîner l'arrêt de la mise en production de la plateforme jusqu'à ce qu'elle soit rendue conforme.

Afin d'évaluer le niveau de conformité et de sécurité de la plateforme \ac{K8S}, l'équipe \ac{SSI} à fait réaliser un 
audit technique par un prestataire externe compétent, ici XMCO, en juillet 2020.
\newline Il nous permis de mettre en évidence les quelques faiblesses de la plateforme et d'obtenir de précieux conseils 
sur les correctifs à appliquer sur la plateforme.

Plusieurs chantiers ont ainsi émergés :
\begin{enumerate}
    \item Cloisonnement de l'infrastructure
    \begin{itemize}
        \item Ségmentation des clusters en fonction des \ac{BU}
        \item Durcissement des règles réseau
    \end{itemize}
    \item Contrôle des déploiements
        \begin{itemize}
        \item Intégration de protection contre objets dangereux
        \item Durcissement des images
        \item Contrôle de l'origine des images
    \end{itemize}
    \item Contrôle d'accès et secrets
    \begin{itemize}
        \item Renforcement du composant kube2iam
        \item Sécurisation du bastion SSH et des serveurs de rebonds
        \item Gestion et rotation des secrets
    \end{itemize}
    \item Validation des objets et Sensibilisation sécurité
    \begin{itemize}
        \item Mise en place de guides pour les développement / déploiement
        \item Sensibilisation des équipes
        \item Standardisation des déploiements
        \item Validation des images et des applications
    \end{itemize}
\end{enumerate}

\pagebreak

\section{La mission et ces objectifs}

Quatres chantiers sont donc à réaliser pour sécuriser la plateforme \ac{K8S} conforme aux politiques et sécurisée.

Deux chantiers ont directement débuté à la suite de l'audit : le cloisonnement de la plateforme ainsi que l'amélioration
du contrôle d'accès. Ces deux chantiers sont réalisés par l'équipe "Études \& Architecture" de la \ac{DSI} Groupe avec
la supervision du \ac{RSSI}.
\newline Il reste donc à traiter le chantier "Contrôle des déploiements" ainsi que le chantier "Validation des objets et 
Sensibilisation sécurité". Ma mission s'inscrit donc dans la réalisation de ces deux chantiers.

Les objectifs de cette mission sont décrits de la façon suivante:
\begin{enumerate}
    \item Objectifs organisationnels : 
    \begin{itemize}
        \item Revue des procédures de qualification sécurité des applications, conteneurs et de déploiements
        \item Revue des procédures de gestion opérationnel de la sécurité des clusters Kubernetes
        \item Production de ressources visant à la standardisation des déploiements sur les clusters
        \item Production de guides et ressources visant à la sécurisation des développements pour Kubernetes
    \end{itemize}
    \item Objectifs techniques:
    \begin{itemize}
        \item Analyse des moyens techniques existants au sein du groupe et évaluation de leur efficacité.
        \item Consolidation et optimisation des moyens techniques existants.
        \item Qualification de nouvelles solutions techniques visant l’amélioration de la sécurité de l’infrastructure Kubernetes du  groupe, 
        tant sur les phases développements que sur les phases de déploiement.
        \item Intégration des solutions retenues à l’infrastructure. 
    \end{itemize}
\end{enumerate}

Ma mission se présentera donc en deux phases distinctes, mêlant chacune des objectifs organisationnels et techniques:
\begin{itemize}
    \item \textmd{Première phase :} Revue et standardisation de la gestion de la sécurité des développements (applicatif et 
    conteneur).
    \item Seconde phase : Mise à disposition d'éléments techniques et / ou opérationnels visant à maintenir un bon 
    niveau de sécurité dans l'exploitation de la plateforme Kubernetes. 
\end{itemize}

