%!TEX root = ../main.tex
\chapter{Introduction}
Cette mission de fin d'étude clôture trois années d'apprentissage réalisées au sein de l'équipe de sécurité informatique 
du Groupe JCDecaux. Le travail réalisé durant les six mois de celles-ci s'inscrivent donc eux aussi dans la continuité 
d'un chantier pluriannuel de modernisation de la chaîne de production des applications du Groupe.  


\section{Le Groupe JCDecaux}
JCDecaux est un groupe industriel international fondé en France en 1964 par Jean-Claude Decaux et spécialisé dans 
l'affichage publicitaire urbain. Principalement connue pour son mobilier urbain (Abribus, MUPI\footnote{Mobilier Urbain 
Pour l'Information - Panneau d'affichage de 2m² dont l'une des faces est réservée pour les collectivités.}, Senior
\footnote{Panneau d'affichage de 4m² en fixation murale ou aérienne.}), l'entreprise familiale a su se hisser en tant 
que leader de la communication extérieure en devenant N°1 mondiale dans ce domaine depuis 2001. 
\newline Son Chiffre d'Affaire pour l'année fiscale 2019 était de 3 890.2 M€.

La notoriété du Groupe JCDecaux s'explique par sa présence dans plus de 80 pays répartis sur tous les continents mais
aussi de par la qualité des services qu'il propose aux collectivités locales et à ses annonceurs.
\newline Cette recherche de l'excellence esthétique et technique pour ses produits fait partie intégrante des 
valeurs de l'entreprise. Ces valeurs sont une réelle fierté pour ses 12 300 collaborateurs et source d'innovation dans leurs
projets. 



\section{L'équipe Sécurité des Systèmes d'Information}
\begin{wrapfigure}{R}{0.27\textwidth} 
    \centering \includegraphics[width=0.25\textwidth]{resources/img/jcd_pla_front.jpg}
    \centering Siège de Sainte Apolline
\end{wrapfigure}
En 2007, la \ac{DSI} Groupe s'est pourvue d'une équipe dédiée à la \ac{SSI}.
Le besoin d'une équipe dédiée devenait en effet de plus en plus urgent suite à l'augmentation des ressources numérique
exploitées par le Groupe; tant pour la diffusion publicitaire programmatique que les services numériques aux 
collectivités (affichage, vélo en libre-service).

L'équipe \ac{SSI} Groupe de JCDecaux est localisée au siège de Sainte-Apolline, Plaisir.
Elle est actuellement composée de trois personnels internes (le \ac{RSSI}, un ingénieur sécurité, un apprenti) et est 
renforcée par trois personnels externes en prestation.

Ces missions s'articulent autour de trois thèmes : la sécurité organisationnelle (Analyse de risques, 
politiques de sécurité, sensibilisation, \dots), la sécurité opérationnelle (\ac{SOC}, conformité, veille phishing, \dots) et 
la qualification technique des systèmes et infrastructures.

Si l'équipe existe depuis plus de dix ans, ses objectifs de sécurisations des \ac{SI} et des services proposés par le 
Groupe JCDecaux ne sont pas encore remplis. Ces objectifs nécessitent en effet un travail permanent de contrôle des 
systèmes, d'accompagnement des développements et de formation des collaborateurs pour atteindre un niveau convenable.

Cela est par ailleurs particulièrement vrai pour les entreprises industrielles pour lesquelles la culture de 
sécurité informatique n'est souvent que peu présente. 
\newline C'est pourquoi les missions réalisées par l'équipe de \ac{SSI} aux profits des filiales ne leur sont pas 
facturés par le Groupe JCDecaux; afin de promouvoir cette culture et favoriser les échanges de compétences dans ce domaine.  



\section{Publicité et développement applicatif}
Contraint par la complexité de son secteur d'activité, le Groupe JCDecaux s'est pourvue en capacité de développement
logiciel et embarqué pour répondre à ses problématiques métier (\eg Booking, préparation, diffusion).
Ses équipes de développement sont réparties dans les différentes \ac{BU} des \ac{DSI} filiale et Groupe.

Depuis 2016, les équipes de développement travaillent à la mise en place de technique agiles et à l'intégration d'une 
méthodologie Agile dans leurs projets. Cet objectif vise à accélérer les développements réalisés 
pour l'entreprise tout en augmentant la flexibilité des applications créés. 

Effet de bords bénéfique à ce changement de doctrine, on constate une augmentation de la fragmentation des applications
(séparation en micro-services et dissociation Frontend applicatif du Backend). Cette fragmentation permet en effet une
simplification des opérations de maintenances et favorise l'amélioration incrémentale des applications produites.

Or, on constate cependant depuis cette migration l'explosion du nombre de serveurs (virtuels ou physique) et une réelle
augmentation de complexité dans l'administration de ces derniers. Une rationalisation des ressources et automatisation
des actions devient donc nécessaire.



\section{Automatisation et conteneurisation du SI}
Pour tenter de palier à ces problématiques d'administration , les équipes de la \ac{DSI} Groupe se sont donc équipées 
de divers outils d'automatisation et méthodes de gestion pour optimiser leurs actions les plus récurrentes.

L'amélioration de la \ac{CMP}, sa liaison avec la \ac{CMDB} et l'intégration d'Ansible \footnote{Ansible est un 
moteur d'automatisation du SI. Il permet l'automatisation de la configuration de système et du déploiement applicatif
grâce à des playbook et est initialement développé par RedHat sous licence GNU GPL v3.0} dans les systèmes permettent aux
équipes opérationnelle de réduire partiellement cette complexité.

Cependant, certaines opérations n'étant pas automatisables, trop complexes ou encore chronophage, nous constatons
occasionnellement des requalifications arbitraires des environnements d'applications métier. Cette culture du 
\emph{POC-PROD}\footnote{Néologisme décrivant le fait de requalifier l'environnement de développement ou d'intégration 
d'un système en tant que production sans migration appropriée ni qualification. \newline POC : Proof of Concept, PROD :
Production}, bien qu'en nette régression, existe encore dans une certaine mesure au sein du Groupe JCDecaux
et pose malheureusement encore des problèmes de sécurité.

En 2019, un programme de qualification de la technologie d'orchestration de conteneur \ac{K8S} est monté 
conjointement entre la \ac{DSI} Groupe et la \ac{DSI} de la filiale France. Il vise, à terme, à fournir aux équipes de 
développement et aux équipes opérationnelles une plateforme de déploiement standardisé d'applications conteneurisées.
\newline Sa mission consiste à évaluer une architecture de cluster \ac{K8S} hébergée sur l'infrastructure de notre Cloud 
Provider \ac{AWS} et à l'intégration des outils de déploiement nécessaires.
\newline Il représente en soi une étape majeure dans l'intégration de la méthodologie DevOps au sein du Groupe JCDecaux
et pourrait grandement réduire le risque d'occurrence de \emph{POC-PROD}.

Cette évaluation maintenant terminée et ses résultats concluants, les responsables du programme souhaitent pouvoir requalifier
la plateforme créer en tant que plateforme de production. Ce souhait est particulièrement motivé par la filiale France
dont les équipes de développements en ont exprimées le besoin.
\newline Cette plateforme devrait à termes être rendu accessible aux autres filiales du Groupe n'ayant pas participé à 
l'expérimentation initiale. Le tout accompagné d'objectifs de rationalisation des ressources et de réduction des coûts.

\newpage

\section{Problématiques de sécurité}
Comme pour tous les projets d'infrastructure informatique du Groupe, la plateforme \ac{K8S} doit respecter un ensemble de
règles et principes d'implémentation définies par les politiques du Groupe JCDecaux. La conformité de la plateforme aux
\emph{IT Security Policies} (\ie Politiques de sécurité informatique en français) est quant à elle un prérequis à tout
requalification en plateforme de production.

Si nous savons qu'un certain nombre de mesures ont été mise en place afin d'assurer l'intégrité et le bon fonctionnement
opérationnelle de la plateforme et que plusieurs éléments de sécurité sont intégrés dans celle-ci, nous ne connaissons pas
le réel niveau de sécurité de la plateforme \ac{K8S}.

Ainsi, afin d'évaluer son niveau de conformité et de sécurité, l'équipe \ac{SSI} a fait réaliser un audit technique par 
un prestataire externe, ici XMCO, en juillet 2020.
\newline Il a permis de mettre en évidence les quelques faiblesses de la plateforme et d'obtenir des conseils 
sur les correctifs à appliquer sur la plateforme.

Plusieurs chantiers de remédiation ont ainsi émergé :
\begin{multicols}{2}
    \begin{enumerate}
        \item Cloisonnement de l'infrastructure
        \begin{itemize}
            \item Ségmentation des clusters en fonction des \ac{BU}
            \item Durcissement des règles réseau
        \end{itemize}
        \item Contrôle d'accès et secrets
        \begin{itemize}
            \item Renforcement du composent kube2iam
            \item Sécurisation du bastion SSH et des serveurs de rebond
            \item Gestion et rotation des secrets
        \end{itemize}
        \columnbreak
        \item Contrôle des déploiements
            \begin{itemize}
            \item Intégration de protection contre objets dangereux
            \item Durcissement des images
            \item Contrôle de l'origine des images
        \end{itemize}
        \item Validation des objets et Sensibilisation sécurité
        \begin{itemize}
            \item Mise en place de guides pour les développements / déploiements
            \item Sensibilisation des équipes
            \item Standardisation des déploiements
            \item Validation des images et des applications
        \end{itemize}
    \end{enumerate}
\end{multicols}

La réalisation de ces quatre chantiers est donc nécessaire afin de pouvoir considérer la plateforme \ac{K8S} comme étant
conforme aux prérequis de sécurité du Groupe et donc pouvoir commencer à l'exploiter en tant que plateforme de production.

Les deux premiers chantiers ont débutés quelques semaines après la restitution de l'audit et ont été encadrés par le \ac{RSSI}.
Ils ne figureront donc pas dans les objectifs de la mission de fin d'étude qui m'a été confié.

\newpage

\section{Ma mission de fin d'études}
Les deux chantiers restants, et non les moindres, portent donc sur la sécurisation et validation des images déployées sur
la plateforme \ac{K8S}, le contrôle de leur déploiement ainsi que la sensibilisation des équipes DevOps.

La réalisation de ces deux chantiers constituera donc l'objectif principal de cette mission de fin d'étude.
Elle pourrait donc être problématisée de la manière suivante : 
\begin{center}
    \color{bluejcd}  \enquote{Comment sécuriser une application conteneurisée, de sa conception à son déploiement sur un
     cluster Kubernetes?}
\end{center}

Nous tacherons donc tout au long de ce mémoire de trouver, évaluer et mettre en place un ensemble de solutions visant
à répondre la problématique formulée ci-dessus.

Nous aborderons dans un premier un ensemble de notions liées au développement applicatif, à la méthodologie DevOps 
et DevSecOps, à la conception de conteneurs et à leur hébergement sur une infrastructure dédiée.

Puis, nous nous concentrerons sur la mise en évidences des risques de sécurité associés au développement 
d'applications conteneurisées. Nous rechercherons par la suite des solutions à ces risques en nous aidant d'une 
méthodologie de référence.

Nous poursuivrons sur l'intégration de ces solutions et évaluerons leur impact sur les équipes exploitant la plateforme
Kubernetes. Nous porterons une attention particulière sur l'évolution, positive ou non, de l'efficacité du processus de 
développement DevOps du Groupe.

Enfin nous conclurons l'adéquation des solutions retenues et de leur intégration dans le contexte DevOps du Groupe JCDecaux.



