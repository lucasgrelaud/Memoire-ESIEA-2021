\chapter{Réalisations techniques}
Comme énoncé dans la première partie de ce mémoire, nus allons concentrer notre travail sur la sécurisation des 
développements applicatifs jusqu'à leur déploiement. Nous aborderons donc des sujets tel que la création de deux 
procédures de sécurité, l'outillage des équipes ainsi que la réalisation de diverses évaluations de sécurité.

\section{Une première approche du DevSecOps}
Rapidement évoqué dans l'état de l'art, l'équipe de \ac{SSI} a entamée ,dès le quatrième semestre de 2018, un premier 
travail d'intégration de sécurité informatique dans les développements au travers de la mise en place d'un processus 
\emph{Security By Design}.

Ce processus, centré en son cœur sur l'accompagnement de \emph{Security Champions}\footnote{Un 
\emph{Security Champion} est un ambassadeur de la \ac{SSI} dans les équipes projets.}, vise à faciliter le questionnement
et  les réflexions sur les sujets de \ac{SSI} dans les équipes projets. Représentés par des responsables produit, des 
développeurs ou opérationnels, ils favorisent les échanges entre les équipes projets et l'équipe de \ac{SSI} en servant 
de réels points de liaisons pour ces dernières.

Grâce au travail des \emph{Security Champions}, le nombre de demande d'audit technique et de conseils d'architecture
a progressivement augmenté tout au long de l'année 2019. L'équipe s'est donc retrouvé à réaliser plusieurs audits par 
mois sur l'année 2019 là où elle n'en réalisait qu'un tous les deux mois en moyenne en 2018. J'ai par ailleurs eu le 
plaisir de m'occuper d'une majorité de ces qualifications sécurité.

C'est d'ailleurs sur cette même période qu'un projet de personnalisations des règles du \ac{SAST} exploité par JCDecaux 
a été lancé. Ces personnalisations, réalisées en deux temps, visaient à créer plusieurs profils d'analyse pour chaque 
langage de programmation utilisé dans l'entreprise, tout en réduisant le bruit généré par des règles remontant trop de 
faut positif.

Grace à ces actions, portés tant auprès du Groupe qu'auprès de ses filiales, l'équipe de \ac{SSI} s'est rapproché un peu
plus de son objectif de sécurisation des applications développées en interne. Cette approche proactive et non plus 
réactive de sécurisation a notamment permis de réduire drastiquement le nombre d'applications rejetées à la suite d'une 
qualifications sécurité pré-déploiement en production.

Nous passions donc d'une méthodologie agile \emph{DevOps} à une méthodologie \emph{DevSecOps}.

\newpage