%!TEX root = ../main.tex
\chapter{Conclusion}
Si ce projet n'est pas tout à fait terminé, un démonstrateur devant être livré et une revue des images devant être faite,
nous pouvons cependant tirer quelques conclusions sur cette mission de fin d'étude et la réalisation du projet.

\section{Réalisation des objectifs}
Les objectifs de cette mission portaient sur la sécurisation des développement dans le cadre d'applications conteneurisées.

Nous devions donc trouve et mettre en œuvre diverses solutions afin de sécuriser dans un premier temps les applications 
développés dans le Groupe JCDecaux, pendant leur développement et leur exploitation.
\newline Puis dans un second temps, nous devions sécuriser le support d'execution de ces applications, 
en l'occurrence des conteneurs fonctionnant sur plusieurs clusters Kubernetes.

Grâce à la refonte du processus d'audit des applications, le Groupe JCDecaux dispose maintenant d'un cadre unique, 
standard et connue de toutes les équipes pour la réalisation de qualification de sécurité des développements internes.
Ce processus, supplémenté d'un accompagnement des équipes dans la mise en place des remédiations aux vulnérabilités 
découvertes, a fait l'objet d'une évaluation en conditions réelles lors d'un audit applicatif.
\newline Les résultats de cette évaluation étant probants, il convient, à mon sens, de considérer que cet objectif est 
atteint.

Si nous ne pouvons encore statuer sur l'atteinte du deuxième objectif, certaines tâches étant toujours en cours de 
réalisation, nous pouvons cependant observer et apprécier les premiers résultats des actions entreprises.

En effet, grâce à la mise en place d'outils d'inventaire et de scanner de vulnérabilités, nous pouvons maintenant 
évaluer le niveau de sécurité des conteneurs fonctionnant sur les clusters Kubernetes du Groupe JCDecaux. Ces nouvelles
capacités permettront un meilleur suivi opérationnel de la sécurité des clusters et de lancer au besoin des plans de 
remédiation, et non plus seulement à la suite d'audit précédent une mise en production.
\newline De plus, grâce au travail de normalisation des conteneurs et à l'outillage des développeurs, nous favorisons 
l'intégration de la gestion des patchs durant les phases de développement, ce qui réduit le risque de blocage de déploiement
et diminue progressivement la dette technique de l'infrastructure.
\newline Les équipes du Groupe JCDecaux se dirigent donc toujours plus vers une méthodologie \emph{DevSecOps} et augmentent 
de ce fait le niveau de sécurité global des infrastructures et applications.

\section{Solution à la problématique}
La problématique à laquelle nous cherchions à répondre tout au long de ce mémoire était la suivante : 
\begin{center}
    \color{bluejcd}  \enquote{Comment sécuriser une application conteneurisée, de sa conception à son déploiement sur un
     cluster Kubernetes?}
\end{center}
Je pense donc maintenant être en mesure d'apporter plusieurs éléments permettant de répondre à cette problématique.

En premier lieu, il est important de comprendre que si les thèmes de la sécurité informatique et du développement sont
avant tout des sujets techniques, l'aspect \emph{humain} n'est pas pour autant à négliger.
\newline En effet, nous aurons bon mettre en œuvre toutes les procédures, les outils et scanner que nous voudrons, si nous
ne somme pas capables de fidéliser les équipes impliquées dans le développement applicatif à notre démarche, nous ne 
pourrons pas sécuriser nos applications et conteneurs.

Un deuxième élément de réponse porterait quant à lui sur la démonstration des \emph{bénéfices} apportés par l'intégration
de la sécurité informatique dans les développements. En effet, si nous ne somme pas de réaliser une intégration efficace
de procédures et outils de sécurité; puis de démontrer leurs bénéfices aux équipes de développements, ces dernières risque,t
de percevoir ces évolutions comme des contraintes et non comme une façon d'améliorer leurs conditions de travail.

Enfin, il est important de rationaliser ces évolutions aux ressources, humaines et techniques, disponible dans l'entreprise
d'accueil. Un projet de sécurisation des développements ne serait être considéré comme viable s'il ne peut assurer un 
impact mesuré sur ces ressources.

\section{Quel futur pour ce projet ?}
Ce projet de sécurisation n'étant pas encore complété, il nous faut d'abord terminer l'ensemble des actions associées.
Cependant, une fois finalisé, nous serons en droit de nous questionner sur les débouchés de ce projet.

Si l'aspect technique et organisationnelle de la sécurité des développements sont portés de façon appropriée dans ce projet,
l'aspect humain et sociétal fait en partie défaut.

En effet, il serait intéressant et bénéfique, tant pour l'entreprise que pour ses employés, de proposer des formations
au développement sécurisé aux équipes internes du Groupe. Le développement et la sécurité informatique étant en constante
évolution, la mise à disposition de plateformes tel que \href{https://www.securecodewarrior.com/}{SecureCodeWarrior} ou 
\href{https://www.hackedu.com/}{HackEDU} permettraient aux équipes d'améliorer leurs compétences et d'effectuer le 
travail de veille nécessaire au développement sécurisé.